%!TEX root = ../draft.tex
\chapter{Motivation}

\section{Ausgangssituation}

Das aktuelle \ac{LWM} wurde für die Praktika des ADV-Labors am Campus Gummersbach der TH-Köln entwickelt. Ziel des Tools ist es, die Massenabfertigung eines Praktikums zu vereinfachen. Dies soll erreicht werden, indem der Staffelplan des Praktikums automatisch generiert wird und im Zuge dessen Konflikte mit anderen Praktika bestmöglich eliminiert werden. Es bietet zudem die Möglichkeit, Praktikumsteilnehmern die Anwesenheit an einem Termin, seine Leistungen sowie die erlangten Bonuspunkte zu vermerken.

\section{Problemstellung}

Da das \ac{LWM} in erster Linie für das ADV-Labor entwickelt wurde, deren Praktika eine gleichbleibende Struktur aufwiesen, wurde das Tool genau für dieses Praktikumsschema modelliert. Mittlerweile hat sich die Struktur der Praktika des ADV-Labors verändert und es nutzen zunehmend mehr Außenstehende Module das LWM. Deshalb ist das Tool zum derzeitigen Stand nicht mehr in der Lage, die Anforderungen der Nutzer optimal zu erfüllen.

\section{Lösungsansatz}

Um herauszufinden welche Funktionalitäten das \ac{LWM} beinhalten muss, damit es den Anforderungen aller Nutzer entspricht, ist ein Verfahren notwendig, welches die Wünsche und Meinungen der Nutzer in die Planung miteinbezieht. Dafür bietet sich eine Anforderungsanalyse an, diese Ergebnisse können dann zur Verbesserung der Nutzbarkeit des Tools beitragen. Mit einer anschließenden Aufwandsschätzung und Priorisierung können die wichtigsten Anforderungen herausgefiltert werden, um sie daraufhin umzusetzen.




\nomenclature{Staffelplan}{Ein Terminplan für das Praktikum in dem alle Termine mit den dazugehörigen Gruppen / Praktikumsteilnehmer eingetragen sind.\vspace{4mm}}

\nomenclature{Modul}{Ein Modul aus dem Modulhandbuch der TH-Köln.\vspace{4mm}}

\nomenclature{Bonuspunkte}{Eine Prüfungsleistung in Punkten, die der Praktikumsteilnehmer während des Praktikums durch seine Leistungen sammeln kann. \vspace{4mm}}































%======================================================================
%======================================================================
% Eintrag ins Glossar
\nomenclature{Stakeholder}{
Ein Stakeholder ist eine Person, Organisation oder Gruppe von Personen, die am Projekt aktiv beteiligt ist oder durch den Projektverlauf oder das Projektergebnis beeinflusst wird (sowohl positiv wie auch negativ), oder die den Projektverlauf oder das Projektergebnis beeinflussen kann (auch dies positiv oder negativ). \citep{GLOSS}
\vspace{4mm}
\label{gloss:stakeholder}
}

\nomenclature{Anforderung}{
Eine Anforderung (engl. Requirement) ist eine Aussage über eine zu erfüllende Eigenschaft (einem Ziel) oder zu erbringende Leistung eines Produktes, Systems oder Prozesses. Also etwas, das das Zielprodukt/-system können muss. \citep{GLOSS}
\vspace{4mm}
\label{gloss:anforderung}
}

\nomenclature{Ziel}{
Ein Ziel bezeichnet einen in der Zukunft liegenden, gegenüber dem Heutigen im allgemeinen veränderten, angestrebten Zustand. Ein Ziel ist somit ein definierter und angestrebeter Endpunkt eines Prozesses. \citep{GLOSS}
\vspace{4mm}
\label{gloss:ziel}
}

\nomenclature{Funktionale Anforderungen}{
Funktionale Anforderungen beschreiben gewünschte Funktionalitäten (was soll das System tun/können) eines Systems bzw. Produkts, dessen Daten oder Verhalten \citep{GLOSS}
\vspace{4mm}
\label{gloss:funktionaleanforderungen}
}

\nomenclature{Qualitätseigenschaften}{
Qualitätseigenschaften beschreiben die "Qualität" in welcher die geforderte Funktionalität zu erbringen ist.
\vspace{4mm}
\label{gloss:qualitätseigenschaft}
}

\nomenclature{Randbedingung}{
Eine Randbedingung ist eine Bedingung die Einfluss auf das Projekt haben kann.
\vspace{4mm}
\label{gloss:randbedingung}
}