%!TEX root = ../draft.tex

  
  	%Erzeugt ein Abbildungsverzeichnis
	\listoffigures
	%Fügt die Zeile "`Abbildungsverzeichnis"' als Chapter ins Inhaltsverzeichnis ein
	\addcontentsline{toc}{chapter}{Abbildungsverzeichnis}
	
	
	%Erzeugt ein Glossar
	\printnomenclature
	%Fügt die Zeile "`Glossar"' als Chapter ins Inhaltsverzeichnis ein
	%\addcontentsline{toc}{chapter}{Glossar}
	
	\chapter*{Abkürzungsverzeichnis}
	\addcontentsline{toc}{chapter}{Abkürzungsverzeichnis}
	\begin{acronym}[FPGA]
	 \setlength{\itemsep}{-\parsep}
	 \acro{LWM}{Labwork Management}
	 \acro{RE}{Requirements Engineering}
	 \acro{MCI}{Mensch-Computer-Interaktion}
	 \acro{US}{User-Story}
	 \acro{SP}{Story-Points}
	 \acro{FPA}{Function-Point-Analyse}
	 \acro{FP}{Function-Points}
	 \acro{SHK}{studentische Hilfskräfte}
	 \acro{WHK}{wissenschaftliche Hilfskräfte}
	\end{acronym}		
		
	%Ändert den Stil des Literaturverzeichnisses
	\bibliographystyle{geralpha}
	%Erzeugt das Literaturverzeichnis anhand der Datei "`literatur.bib"'
	\pagebreak
	\addcontentsline{toc}{chapter}{Literaturverzeichnis}
	\bibliography{literatur}
	%Fügt die Zeile "`Literaturverzeichnis"' als Chapter ins Inhaltsverzeichnis ein
  
  	
\nocite{*}