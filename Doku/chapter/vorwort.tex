%!TEX root = ../draft.tex
\chapter*{Vorwort}
\addcontentsline{toc}{chapter}{Vorwort}

Vor Ihnen liegt das Praxisprojekt \glqq Anforderungsanalyse für ein Praktikums-Management-Werkzeug zum Einsatz in heterogenen Laborstrukturen\grqq . Die in dieser Dokumentation beschriebene Forschungsarbeit wurde in enger Zusammenarbeit mit dem Entwickler-Team des LWM durchgeführt, in dem auch ich seit einem Jahr als Backend-Entwickler tätig bin.\\

Für das entgegengebrachte Engagement möchte ich meinen Kollegen Uwe Müsse und
Alexander Dobrynin danken. Ohne ihr Zutun wäre ein großer Teil des hier erreichten Ergebnisses nicht in dieser Form zustande gekommen. Auch haben sie mir in schwierigen Situationen Hilfestellungen und Denkansätze gegeben.\\

Im Laufe des Projekts wurden zahlreiche Interviews mit Modulverantwortlichen und ihren Mitarbeitern geführt. Diese Gespräche haben im Schnitt 1 - 1,5 Stunden gedauert, trotzdem haben mir die Beteiligten bereitwillig zur Verfügung gestanden, weshalb auch ihnen mein Dank gilt. Ohne sie wäre der Kern dieser Arbeit nicht realisierbar gewesen.\\

Ein weiterer Dank gebührt meiner Freundin, die mir während den Abschlussarbeiten den Rücken frei gehalten hat. Nur so war es mir Möglich, diese rechtzeitig fertigzustellen.\\

Ich wünsche Ihnen viel Spaß beim Lesen.\\

Florian Herborn