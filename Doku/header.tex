\documentclass[BCOR=12mm,DIV11,titlepage,a4paper,oneside]{scrbook}

%Paket für deutsche Silbentrennung etc.
\usepackage{ngerman}

%Paket für Zeichenkodierung, entspricht UTF-8
\usepackage[utf8x]{inputenc}

%Paket das die Ausgabefonts definiert
\usepackage[T1]{fontenc}

%Paket für Sonderzeichen wir RightsReserved
\usepackage{textcomp}

%Euro-Symbol
\usepackage[right]{eurosym}

% Matheumgebung
\usepackage[]{framed, amsmath}

%Paket für das Einbinden von Grafiken über die figure-Umgebung
\usepackage{graphicx}

%Paket zum Ändern der Kopf- und Fußzeile
\usepackage{fancyhdr}
%Benutzt das Paket für eigenen Seitenstil
\pagestyle{fancy} 
%Erzeugt eine Linie in der Kopfzeile (lässt sich mit 0.0pt ausblenden)
\renewcommand*{\headrulewidth}{0.8pt} 
\renewcommand*{\headrule}{\hbox to\headwidth{%
    \color{thmagenta}\leaders\hrule height \headrulewidth\hfill}}
\fancyhf{}
\fancyhead[EC,OC]{\thepage}
% \fancyhead[EL]{\leftmark} 
% \fancyhead[OR]{\rightmark} 
% \fancyhead[ER,OL]{\thepage}
\renewcommand{\sectionmark}[1]{ 
\markboth{\thechapter{} #1}{\thechapter{} #1} 
}
% Ermöglicht ToDos ansprechend zu setzen
\usepackage{xargs} 
\usepackage[colorinlistoftodos,prependcaption,textsize=tiny]{todonotes}
\newcommandx{\unsure}[2][1=]{\todo[linecolor=red,backgroundcolor=red!25,bordercolor=red,#1]{#2}}
\newcommandx{\change}[2][1=]{\todo[linecolor=blue,backgroundcolor=blue!25,bordercolor=blue,#1]{#2}}
\newcommandx{\info}[2][1=]{\todo[linecolor=OliveGreen,backgroundcolor=OliveGreen!25,bordercolor=OliveGreen,#1]{#2}}
\newcommandx{\improvement}[2][1=]{\todo[linecolor=Plum,backgroundcolor=Plum!25,bordercolor=Plum,#1]{#2}}
\newcommandx{\thiswillnotshow}[2][1=]{\todo[disable,#1]{#2}}

%Ändert die Seitennummerierung beim Inhaltsverzeichnis mit eigenem Stil
\renewcommand*{\indexpagestyle}{fancy}
%Verhindert die Seitennummerierung auf den Part-Seiten
\renewcommand*{\partpagestyle}{empty}
%Ändert die Seitennummerierung bei Chapter mit eigenem Stil
\renewcommand*{\chapterpagestyle}{fancy}

%Abbildungsnummerierung ändern (abhängig von chapter, z.B. Abbildung 1.1)
\renewcommand*{\thefigure}{\thechapter.\arabic{figure}}
%Tabellennummerierung ändern (abhängig von chapter, z.B. Tabelle 1.1)
\renewcommand*{\thetable}{\thechapter.\arabic{table}}

%Paket, um ein Glossar/Abkürzungsverzeichnis anzulegen
\usepackage[intoc]{nomencl}
\let\abbrev\nomenclature
%Der Name wird in Glossar geändert
\renewcommand{\nomname}{Glossar}
%Definiert die Aufteilung im Glossar zwischen Begriffen und Erläuterung
\setlength{\nomlabelwidth}{.25\hsize}
%Definiert die Punktelinien im Glossar
\renewcommand{\nomlabel}[1]{#1 \dotfill}
\setlength{\nomitemsep}{-\parsep}
%Veranlasst die Erstellung des Glossars
\makenomenclature

%Einrückungen nach Absätzen und Grafiken verhindern
\setlength{\parindent}{0pt}

\usepackage[nohyperlinks, printonlyused, withpage, smaller]{acronym}


%Verhindern, dass eine neue Seite für ein einzelnes Wort/Zeile verwendet wird
\clubpenalty = 10000 % schliesst Schusterjungen aus 
\widowpenalty = 10000 % schliesst Hurenkinder aus (keine Beleidigung, sondern wirklich ein Fachbegriff)

%Paket für ein deutsches Literaturverzeichnis
\usepackage[authoryear,round]{natbib}
\bibliographystyle{natdin}
% \setlength\bibhang{30pt}

%Paket für die Verwendung von URLs durch den Befehl \url{}
\usepackage{url}

%Paket für Zeilenabstand (onehalfspace, singlespace)
\usepackage{setspace}

%Paket zur Erzeugung von Anführungszeichen durch \enquote{Text}
\usepackage[ngerman]{babel}
\usepackage[babel, german=quotes]{csquotes}

%Paket für farbigen Text
%black,white,green,red,blue,yellow,cyan,magenta
\usepackage{color}
%Farbige Tabellen
\usepackage{colortbl}

%Rotation von Gleitobjekten (Grafiken, Trabellen, etc.)
\usepackage{rotating}

%Rotation von einzelnen Seiten begin{landscape}
\usepackage{lscape}

%Paket für farbigen Hintergrund für Verbatim-Umgebung (Quelltext-Umgebung)
\usepackage{fancyvrb}
\usepackage{verbatim,moreverb}
%Grauton für Quelltext-Umgebung definieren 80% Grau
\definecolor{sourcegray}{gray}{.80}

%Definition der TH Köln-Farben
\definecolor{thred}{cmyk}{0.0,1.0,1.0,0.15}
\definecolor{thorange}{cmyk}{0.0,0.75,1.0,0.0}
\definecolor{thmagenta}{cmyk}{0.3,0.95,0.0,0.0}
\definecolor{thblack}{cmyk}{0.0,0.0,0.0,1.0}

%Paket für Quelltext-Umgebung
\usepackage{listings}
%Alternative Quelltext-Umgebung
%\lstset{numbers=left, 
%	numberstyle=\tiny, 
%	numbersep=5pt,
%	language=Java,
%	breaklines=true,
%	breakautoindent=true,
%	postbreak=\space,
%	tabsize=2,
%	frame=tlrb,
%	basicstyle=\ttfamily\footnotesize}

\usepackage{tikz}

\tikzstyle{sourcecodebox} = [
    draw=blue, very thick,
    rectangle, rounded corners,
    inner sep=10pt
]
\tikzstyle{sourcecodetitle} = [
    fill=black, text=white,
    rectangle, rounded corners
]

\makeatletter
\lstnewenvironment{javacode}[1][]{%
    \def\javacodetitle{#1}%
    \lstset{%
        language=java,
        basicstyle=\ttfamily\footnotesize,
        escapeinside={(*@}{@*)},
        %numbers=left,
        breaklines=true,
        breakatwhitespace=true,
        showspaces=false,
        showstringspaces=false,
        frame=shadowbox,
        frameround=rrrt,
        keywordstyle=\color{blue}\ttfamily,
        commentstyle=\color{comment}\ttfamily,
        linewidth=.95\textwidth,
        rulecolor=\color{black},
        rulesepcolor=\color{gray}
    }%
    \setbox\@tempboxa=\hbox\bgroup\color@setgroup
}%
{%
    \color@endgroup\egroup
    \begin{tikzpicture}
        \node[sourcecodebox] (box)
            % Makebox is needed to take the frame added by listings into account
            {\makebox[.95\textwidth][l]{\box\@tempboxa}};
        \node[sourcecodetitle] at (box.north) 
        {\javacodetitle};
    \end{tikzpicture}
}

% Implementierung innerhalb des Dokumentes
%\begin{javacode}[Initialisierung AlchemyAPI]
%public static AlchemyAPI alchemyObj;
%
%public static void init() {
%    alchemyObj = AlchemyAPI.GetInstanceFromString(KEY);
%}
%
%\end{javacode}

%Paket für Positionierung der Objekte ohne Float (Verwendungsbsp.: \begin{figure}[H])
%\usepackage{here}
%Alternatives Paket für here.sty
\usepackage{float}

%DRAFT als Wasserzeichen im Hintergrund
% \usepackage{draftwatermark} 
% \SetWatermarkAngle{60}
% \SetWatermarkScale{5.0}

%Für lange Tabellen
\usepackage{longtable,array,supertabular}

%Für rowspan in Tabellen
\usepackage{multirow}

%Behält die Schriftgröße der Überschrift normal, wenn z.B. die Schriftgröße in einer Tabelle verändert wird
\addtokomafont{caption}{\normalsize} 

%Paket um PDF Seiten einzubinden
\usepackage{pdfpages}

%Paket zur Erzeugung von Hyperrefs und PDF Informationen
\usepackage[pdftex,plainpages=false,pdfpagelabels,
            pdftitle={title},
            pdfauthor={name}
            ]{hyperref}
%Farben für Links
%Farbige Ränder bei false und farbige Texte bei true
\hypersetup{colorlinks=true,citecolor=black,filecolor=black,linkcolor=black,urlcolor=black}

\apptocmd{\UrlBreaks}{\do\f\do\m}{}{}

%Zwei Verzeichnisse für Inhalt und Anhang
\usepackage{appendix}
% \usepackage{minitoc} 
% \nomtcrule
% \renewcommand{\mtctitle}{Anhangsverzeichnis}
% \setlength{\mtcindent}{0pt}