\documentclass[a4paper]{article}

%Paket für UTF-8
\usepackage[utf8]{inputenc}

%Paket für deutsche Silbentrennung etc.
\usepackage{ngerman}

\begin{document}
\title{
	%Title
	Planung eines erweiterbaren Frameworks zur Erstellung virtueller Welten \\
	%Subtitle
  \large Praxisprojekt 2017
	}
\author{Florian Herborn}
\maketitle


\section{Projektidee}
In meinem Praxisprojekt mache ich es mir zur Aufgabe, ein erweiterbares Framework zur Erstellung virtueller Welten in Form eines Unity3D-Plugin zu planen, um dieses in der Bachelorarbeit weiter zu führen.\\
Dazu ist es notwendig, Informationen über die Generierung virtueller Welten zu sammeln und das Framework bestmöglich auf die erforderlichen Anforderungen vorzubereiten.\\ 
Auch soll es möglich sein, dass Anwender mit nur wenig Aufwand das System und somit seine Welten-Generierung individualisieren, detaillieren und/oder erweitern können. Dazu müssen Berechnungsvorgänge und Informationsfluss in angemessenem Maß dynamisch gestaltet und eigene Module einbindbar sein. \\
Das System soll so gestaltet sein, dass Teil-Berechnungen neu angestoßen und alle davon abhängigen Berechnungen neu kalkuliert werden können. 


\section{Projektverlauf}
Beginnen soll das Projekt am 01.03.2017. Die Fertigstellung erfolgt voraussichtlich am 01.06.2017. Hier geht es dann in die Bachelorarbeit über. \\
In den ersten 4 Wochen werde ich mich voraussichtlich mit der Informations- und Anforderungsfindung beschäftigen, bestehende Systeme begutachten und bekannte Techniken und Verfahren ermitteln. In den darauf folgenden 4 Wochen werde ich das Framework mit den gewonnenen Erkenntnissen designen.\\
Die restliche Zeit ist für Restarbeiten, Überarbeitung und Dokumentation vorgesehen. 


\section{Projektumgebung}
Das Projekt wird als Unity3D-Plugin entworfen. Die Visualisierung wird mit Hilfe der Unity3D-Editorgui realisiert. Die Scripte werden in C\# umgesetzt, zur Vereinfachung werden Unity3D-Datenstrukturen wie Vektoren und deren Berechnungen verwendet.

\end{document}